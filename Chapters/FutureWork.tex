\chapter{Conclusions}

We have examined various approaches to plagiarism detection in reasearch,
and how they are applicable to the idea of similarity detection.
Highlighting an area that is somewhat underrepresented in the literature,
we created a system to perform similarity analysis on Java code based on a 
parse tree kernel approach. The algorithm has proven problematic and needed
to be modified before it could be implemented in such a manner.
However once these challenges were overcome, we were able to produce a tool
that was able to give us genuine insight into the code base we were comparing.
Not completely perfect, the implementation still fails on some more complex
projects and could be refined further to deal with these.

We have also been able to integrate this with Eclipse in a plugin, providing
easy access to our tool for any potential users, and ease of analysis within
the plugin by interacting directly with the source code's AST. A view has
also been created to compare the code side-by-side to assist in the code
examination process. This comparison view is something we would have liked
to explore a lot more, but the complexity of dealing with the Eclipse 
source code, in particular extending classes with package protected variables,
proved time consuming. In the end, the functional if ugly comparison view
is quite an achievement.

\chapter{Future Work}
\label{Future Work}

The research undertaken in this project and work completed has opened up
numerous avenues for further exploration in this field. Some of these
relate specifically to the plugin developed, such as speed improvements and extensions.
There is also scope to expand on the theoretical aspect of the project, with potential
new algorithms or analyses to be presented.

\subsection{Speed up via Heuristics}
\begin{itemize}
\item
Much of the algorithm implementation is naive with respect to the problem domain.
Although we have looked at an approach which uses the code's parse tree to 
analyse the structure, other factors that we know to be true have been omitted
in our solutions. 

Java is a fully object oriented language, which informs on a number of details
about the files we are analysing. The simplest of these to catch is the file name;
when iterating over the projects' files, our implementation compares every file
in the first project with every file in the second. Given that the file names
are the same as the class names, it may be the case that files of the same name
likely represent the same class, and so the similarity of the two can be directly
calculated. This could massively speed up comparison time where projects generally
share the same structure. Though this would be trivial to implement now, a proper
analysis would need to be performed to weigh up the speed benefit with the loss in
accuracy that would occur: if classes were split across multiple files, or named
poorly to clash with other, non-similar classes, the speed increase would likely not
be worth it. Testing this on real world data with student submissions would be
a good method for determining this.

\item
When comparing a certain file with others, skipping files with a low similarity 
to already matched files (either from the same or different projects) provides
an opportunity to get a speed increase from our algorithm. The logic is that
similarity is thought to be transitive; if File A is similar to File B, and
File B is dissimilar to File C, it follows that File A would likely be dissimilar
to File C. Again, whilst this would be fairly straightforward to implement, it
is trickier to be certain of the advantages this would bring. As discussed in
\cref{Evaluation}, there can be multiple clusters that a piece of code belongs
to, in the case described, applying this method may result in the cluster
spanning files skipping comparisons with files that actually prove to be similar.
\end{itemize}

\subsection{Correctness Improvements}

\begin{itemize}
\item
The algorithm already performs fairly efficiently, given a large number of 
projects; however it has been demonstrated that it can often be lacking
in correctness, especially for the more complex projects. To improve this,
there are a number of approaches that could be taken. One interesting idea
touched upon previously is the increasing of dimensions in similarity clustering.
It would be interesting to see how the code is able to cluster when it isn't forced
into a 2 dimensional grid, but rather could cluster organically. It is likely
that the correctness of the clustering would increase, however it is also possible 
that an increase in dimensionality would hamper readability and the similarity
clusters would thus be difficult to spot.

\item
Programmers often have their own style of coding. Examples include
declaration of variable in vs. outside of for loop declarations;
using the ternary operator vs. an if-then-else. These language
different consructs perform the same function as their alternatives,
however would come under a different node type during comparison
and thus be seen as not a match, despite the code being similar as a result of
this. If the nodes were transformed such that the functionally equivalent
styles were made identical, it would hepl improve useful feedback by
categorising the users as similar.

\item
Much has been made of the other types of algorithms in \cref{Background}, and
their robustness and correctness has been well proven. A merged approach of types
of similarity measure could be used to inform on each other how similar certain
constructs are. A learning approach could even be used so that the separate algorithms
could learn from each other, improving their own correctness.
\todo{integrate with other algorithms}
\end{itemize}
% \section{Additions}

% \todo{other languages with CDT etc.}

% \todo{improvements to compare view}

% \todo{different similarity algos?}

% \todo{modification to show difference? if they dont match return 1, else return 1-(1/depth+1)*lambda * min(recurse)}
\chapter{Background}

\label{Background}

\lhead{Chapter 2. \emph{Background}}

\section{Plagiarism Detection}

Though our primary aim is to provide insightful analysis of the students' code 
submissions and not to detect plagiarism, a lot of our research actually took 
place in the realm of plagiarism. In detecting plagiarism, programs are attempting
to output a binary result of ``plagiarism detected'' or ``no plagiarism detected''.
To accomplish this, however, the programs must first build a representation of 
how similar the code is, and then flag code which is similar above a certain threshold.
It is now clear why we should leverage this research -- to reliably detect plagiarism,
we must first reliably judge similarity.

Code similarity detection techniques have been divided up into a number of subclasses:
Text/token, source tree, PDG (Program Dependency Graph) and metric based[TODO: cite the Roy paper here]. 
For our pruposes, the PDG approach seems least useful, as souce code in student exercises
tends to be fairly small scale; there isn't much dependencies between different
packages as the code, in general, is only based around a small number of packages.

\section{Eclipse plugin}

The Eclipse IDE is a development environment for the Java programming language;
It provides tools for Java developers to write compile and debug their applications.
Further to this, it also provides a feature rich environment in which developers
can create additional plugins to add functionality to the base Eclipse distribution.

With access to this plugin framework, Eclipse exposes some of it's internal APIs
and allows developers to be able to manipulate Java in complex ways. The key feature
of this, as it relates to us, is the ability to plug in to the Java compiler of Eclipse
and manipulate the Parse trees as we wish. This easy access to a parse tree in a
feature rich environment such as Eclipse was our main motivation behind the approach
to our primary application. Utilising the plugin environment, we could examine the
source code as it is parsed, and perform our comparison, all within the standard
compilation, with no external tools needed.

TODO: orange
